\documentclass[10pt,landscape,letterpaper,notitlepage]{article}

\usepackage[utf8]{inputenc}
\usepackage[T1]{fontenc}
\usepackage[spanish,activeacute]{babel}
\usepackage{geometry}
\usepackage{listings}
\usepackage{fancyhdr}
\usepackage{lastpage}
\usepackage{amssymb}
\usepackage[usenames,dvipsnames]{color}
\usepackage{graphicx}
\usepackage{wrapfig}
\usepackage{amsmath}
\usepackage{makeidx}
\usepackage{multicol}
\usepackage{hyperref}

%%% Márgenes para aprovechar toda la hoja horizontal
\geometry{
    paper=letterpaper,
    landscape,
    top=1cm,
    bottom=2cm,
    left=2cm,
    right=1.5cm,
    columnsep=1cm
}

%%% Encabezado y pie de página
\pagestyle{fancy}
\fancyhead{} % limpiar encabezado
\fancyfoot[C]{Pagina \thepage\ de \pageref{LastPage}} % pie de página centrado
\renewcommand{\headrulewidth}{0pt}
\renewcommand{\footrulewidth}{0.4pt}

%%% Colores para Listings
\definecolor{myred}{rgb}{0.8,0,0}      % rojo
\definecolor{mygreen}{rgb}{0,0.6,0}    % verde
\definecolor{mylila}{rgb}{0.6,0,0.6}   % lila/morado


\lstloadlanguages{C++}
\lstset{
    language=C++, 
    basicstyle=\ttfamily\normalsize,  % ligeramente más grande
    keywordstyle=\color{myred}\ttfamily,
    stringstyle=\color{magenta}\ttfamily,
    commentstyle=\color{mygreen}\ttfamily,
    morecomment=[l][\color{mygreen}]{\#},
    numbers=left,
    numberstyle=\tiny,
    stepnumber=1,
    numbersep=5pt,
    frame=l,
    framesep=2pt,
    breaklines=true,
    breakatwhitespace=false,
    columns=fixed,
    linewidth=\columnwidth,
    showstringspaces=false,
    tabsize=2,
    flexiblecolumns=false,
    lineskip=0pt
}

%%% Macros
\def\nbtitle#1{\begin{Large}\begin{center}\textbf{#1}\end{center}\end{Large}}
\def\nbsection#1{\section{#1}}
\def\nbsubsection#1{\subsection{#1}}
\def\nbcoment#1{\begin{small}\textbf{#1}\end{small}}
\newcommand{\comb}[2]{\left( \begin{array}{c} #1 \\ #2 \end{array}\right)}
\def\complexity#1{\texorpdfstring{$\mathcal{O}(#1)$}{O(#1)}}
\newcommand{\cppfile}[2]{\lstinputlisting[style=C++,linerange={#1}]{#2}}

\begin{document}

\nbtitle{HandBook de romerproblem}

\begin{multicols*}{2}

\section*{Math}

\subsection*{matriz_fibo}
\begin{lstlisting}
struct matrix {
    long long mat[2][2];
    matrix friend operator *(const matrix &a, const matrix &b){
        matrix c;
        for (int i = 0; i < 2; i++) {
          for (int j = 0; j < 2; j++) {
              c.mat[i][j] = 0;
              for (int k = 0; k < 2; k++) {
                  c.mat[i][j] += a.mat[i][k] * b.mat[k][j];
              }
          }
        }
        return c;
    }
};

matrix matpow(matrix base, long long n) {
    matrix ans{ {
      {1, 0},
      {0, 1}
    } };
    while (n) {
        if(n&1)
            ans = ans*base;
        base = base*base;
        n >>= 1;
    }
    return ans;
}

ll fib(int n) {
    matrix base{ {
      {1, 1},
      {1, 0}
    } };
    return matpow(base, n).mat[0][1];
}\end{lstlisting}

\section*{Miscelanea}

\section*{Data Structures}

\subsection*{fenwickTree}
\begin{lstlisting}
#include <bits/stdc++.h>

using namespace std;
typedef long long ll;


struct Bit
{
    private:
        vector<ll> bit;
        int sz;
    public:
    
    Bit(int n){bit.resize(n+1,0);this->sz=n;}

    void init(vector<ll> vec)
    {
        for(int i=0;i<sz;i++)
        {
            update(i+1,vec[i]);
        }
    }

    void update(int p,ll val)//1i
    {
        while(p<=sz)
        {
            bit[p]+=val;
            p+=(p&(-p));
        }
    }
    ll query(int p)//[1,p]1i
    {
        ll ans=0;
        while(p>0)
        {
            ans+=bit[p];
            p-=(p&-p);
        }
        return ans;
    }

    ll rquery(ll left,ll right)//[l,r]1i
    {
        return query(right)-query(left-1);
    }
};
\end{lstlisting}

\subsection*{lca}
\begin{lstlisting}
\end{lstlisting}

\subsection*{segTree}
\begin{lstlisting}
#include <bits/stdc++.h>

using namespace std;
typedef long long ll;

struct Data
{
    ll cant = 0;
    Data() { cant = 1e18; }

    Data(ll c) { cant = c; }
};

struct SegTree
{
private:
    vector<Data> st;

public:
    int sz;
    Data merge(Data a, Data b)
    {
        return min(a.cant, b.cant);
    }
    void init(int n, vector<Data> v)
    {
        while (__builtin_popcount(n) != 1)
        {
            n++;
        }
        st.resize(2 * n, Data());
        sz = n; // solo n, NO 2*n
        for (int i = 0; i < (int)v.size(); i++)
        {
            st[n + i] = v[i];
        }

        for (int i = n - 1; i > 0; --i)
        {
            st[i] = merge(st[i << 1], st[(i << 1) + 1]);
        }
    }

    void updateTreeNode(int p, Data nuevoValor) // 0i pos
    {
        st[p + sz] = nuevoValor;
        p = p + sz;
        for (int i = p; i > 1; i >>= 1)
            st[i >> 1] = merge(st[i], st[i ^ 1]);
    }

    Data query(int nodo, int left_nodo, int right_nodo, int l_q, int r_q) //
    {
        // query(1,0,st.sz-1,l_q,r_q) tipo->[l_q,r_q]0i
        /*
         los indices de los nodos empieza desde 1;
         */
        if (l_q <= left_nodo && right_nodo <= r_q)
        {
            return st[nodo];
        }
        if (l_q > right_nodo || left_nodo > r_q)
        {
            return Data();
        }

        int mitad = (left_nodo + right_nodo) / 2; ///[l:r] --> [l:mitad] [mitad+1:r]
        return merge(query(nodo * 2, left_nodo, mitad, l_q, r_q), query(nodo * 2 + 1, mitad + 1, right_nodo, l_q, r_q));
    }
};
\end{lstlisting}

\section*{Strings}

\subsection*{hashing}
\begin{lstlisting}
struct StrHash
{ // Hash polinomial con exponentes decrecientes.
    static constexpr ll ms[] = {1'000'000'007, 1'000'000'403};
    static constexpr ll b = 500'000'000;
    vector<ll> hs[2], bs[2];
    StrHash(string const &s)
    {
        int n = s.length();
        for(int k=0;k<2;k++)
        {
            hs[k].resize(n + 1), bs[k].resize(n + 1, 1);
            for(int i=0;i<n;i++)
            {
                hs[k][i + 1] = (hs[k][i] * b + s[i]) % ms[k];
                bs[k][i + 1] = bs[k][i] * b % ms[k];
            }
        }
    }
    ll get(int idx, int len) const
    { // Hashes en `s[idx, idx+len)`.
        ll h[2];
        for(int k=0;k<2;k++)
        {
            h[k] = hs[k][idx + len] - hs[k][idx] * bs[k][len] % ms[k];
            if (h[k] < 0)
                h[k] += ms[k];
        }
        return (h[0] << 32) | h[1];
    }
};

//concate substrings(or strings) from non necesary two differents strings [idx,indx+lex)
ll concat_cross_hashes(const StrHash& A, int i1, int len1, const StrHash& B, int i2, int len2)
{
    ll res[2];
    for (int k = 0; k < 2; ++k)
    {
        // hash de substring A[i1..i1+len1-1]
        ll h1 = A.hs[k][i1 + len1] - A.hs[k][i1] * A.bs[k][len1] % A.ms[k];
        if (h1 < 0) h1 += A.ms[k];

        // hash de substring B[i2..i2+len2-1]
        ll h2 = B.hs[k][i2 + len2] - B.hs[k][i2] * B.bs[k][len2] % B.ms[k];
        if (h2 < 0) h2 += B.ms[k];

        // combinación: h1 * b^len2 + h2
        res[k] = (h1 * A.bs[k][len2] + h2) % A.ms[k];
    }
    return (res[0] << 32) | res[1];
}


//0 indexed
ll h=StrHash("Hola").get(0,0+"Hola".size());
\end{lstlisting}

\subsection*{hashing2d}
\begin{lstlisting}
struct Hashing
{
    vector<vector<int>> hs;
    vector<int> PWX, PWY;
    int n, m;
    static const int PX = 3731, PY = 2999, mod = 998244353;
    Hashing() {}
    Hashing(vector<string> &s)
    {
        n = (int)s.size(), m = (int)s[0].size();
        hs.assign(n + 1, vector<int>(m + 1, 0));
        PWX.assign(n + 1, 1);
        PWY.assign(m + 1, 1);
        for (int i = 0; i < n; i++)
            PWX[i + 1] = 1LL * PWX[i] * PX % mod;
        for (int i = 0; i < m; i++)
            PWY[i + 1] = 1LL * PWY[i] * PY % mod;
        for (int i = 0; i < n; i++)
        {
            for (int j = 0; j < m; j++)
            {
                hs[i + 1][j + 1] = s[i][j] - 'a' + 1;
            }
        }
        for (int i = 0; i <= n; i++)
        {
            for (int j = 0; j < m; j++)
            {
                hs[i][j + 1] = (hs[i][j + 1] + 1LL * hs[i][j] * PY % mod) % mod;
            }
        }
        for (int i = 0; i < n; i++)
        {
            for (int j = 0; j <= m; j++)
            {
                hs[i + 1][j] = (hs[i + 1][j] + 1LL * hs[i][j] * PX % mod) % mod;
            }
        }
    }
    int get_hash(int x1, int y1, int x2, int y2)
    { // 1-indexed
        assert(1 <= x1 && x1 <= x2 && x2 <= n);
        assert(1 <= y1 && y1 <= y2 && y2 <= m);
        x1--;
        y1--;
        int dx = x2 - x1, dy = y2 - y1;
        return (1LL * (hs[x2][y2] - 1LL * hs[x2][y1] * PWY[dy] % mod + mod) % mod -
                1LL * (hs[x1][y2] - 1LL * hs[x1][y1] * PWY[dy] % mod + mod) % mod * PWX[dx] % mod + mod) %
               mod;
    }
    int get_hash()
    {
        return get_hash(1, 1, n, m);
    }
};
\end{lstlisting}

\section*{Segment Tree}

\subsection*{cantidad_minimos}
\begin{lstlisting}


#include <bits/stdc++.h>
using namespace std;
typedef vector<int> vi;

struct info
{
    int num;
    int cant=1;
};

info merge(info a, info b)
{
    if(a.num<b.num)
    {
        return a;
    }
    else if(a.num>b.num)
    {
        return b;
    }
    else
    {
        info res;
        res.num = min(a.num, b.num);
        res.cant=a.cant+b.cant;
        return res;
    }
    
}

vector<info> st;
int n;

void buildSt(vi vec)
{
    for (int i = 0; i < n; i++) // Añadimos las hojas
    {
        info a;
        a.num = vec[i];
        st[n + i] = a;
    }
    for (int i = n - 1; i > 0; --i) // construmos los padres a partir de las hojas
    {
        st[i] = merge(st[i << 1], st[(i << 1) + 1]);
    }
}

void updateTreeNode(int p, info nuevoValor)
{
    // set value at position p
    st[p + n] = nuevoValor;
    p = p + n;

    // actualizar padres
    for (int i = p; i > 1; i >>= 1)
        st[i >> 1] = merge(st[i], st[i ^ 1]);
}

info query(int nodo, int left_nodo, int right_nodo, int l_q, int r_q)
{
    /*
     los indices de los nodos empieza desde 1;
     */
    if (l_q <= left_nodo && right_nodo <= r_q)
    {
        return st[nodo];
    }
    if (l_q > right_nodo || left_nodo > r_q)
    {
        info xd;
        xd.num = 1e9 + 5;
        return xd; // neutro
    }

    int mitad = (left_nodo + right_nodo) / 2; ///[l:r] --> [l:mitad] [mitad+1:r]
    return merge(query(nodo * 2, left_nodo, mitad, l_q, r_q), query(nodo * 2 + 1, mitad + 1, right_nodo, l_q, r_q));
}

int main()
{
    cin >> n;
    int Q;cin>>Q;
    vector<int> vec(n);
    for (int i = 0; i < n; i++)
    {
        cin >> vec[i];
    }

    while (__builtin_popcount(n) != 1)
    {
        vec.push_back(1e9 + 50); // introduce neutro;
        n++;
    }
    st.resize(2 * n);

    buildSt(vec); // construimos el st

    
    while(Q--)
    {
        int q;cin>>q;
        if(q==1){ //update, son 0 indexed

            int indi;cin>>indi;
            info val;cin>>val.num;
            updateTreeNode(indi,val);

        }
        else
        {
            int l,r;cin>>l>>r;
            //q(nodo inicio,rango inicio 0i,rango fin0i,q_inicio 0i,q_fin 0i)

            info ans=query(1,0,n-1,l,r-1);//la resta a l o r denpenda del input constrait
            cout<<ans.num<<" "<<ans.cant<<endl;

        }
    }

    

    return 0;
}\end{lstlisting}

\subsection*{max_segment_sum}
\begin{lstlisting}


#include <bits/stdc++.h>
using namespace std;
typedef vector<int> vi;
typedef long long ll;

const ll NEG_INF = -1e15;

struct info
{
    ll sum;
    ll maxSuff;
    ll maxPreff;
    ll ans;
};

info make_data(ll num)
{
    info ret;
    ret.sum = num;
    ret.maxPreff = ret.maxSuff = ret.ans = max(0LL, num);  // <= permite segmento vacío
    return ret;
}

info merge(info a, info b)
{
    info ret;
    ret.sum = a.sum + b.sum;
    ret.maxPreff = max({a.maxPreff, a.sum + b.maxPreff});
    ret.maxSuff = max({b.maxSuff, b.sum + a.maxSuff});
    ret.ans = max({a.maxSuff + b.maxPreff, a.ans, b.ans});
    return ret;
}

info neutral()
{
    return {0, 0, 0, 0};
}

vector<info> st;
int n;

void buildSt(vector<ll> vec)
{
    for (int i = 0; i < n; i++) // Añadimos las hojas
    {
        if (vec[i] == NEG_INF)
        {
            st[n + i] = neutral();
        }
        else
        {
            info a = make_data(vec[i]);
            st[n + i] = a;
        }
    }
    for (int i = n - 1; i > 0; --i) // construmos los padres a partir de las hojas
    {
        st[i] = merge(st[i << 1], st[(i << 1) + 1]);
    }
}

void updateTreeNode(int p, info nuevoValor)
{
    // set value at position p
    st[p + n] = nuevoValor;
    p = p + n;

    // actualizar padres
    for (int i = p; i > 1; i >>= 1)
        st[i >> 1] = merge(st[i], st[i ^ 1]);
}

info query(int nodo, int left_nodo, int right_nodo, int l_q, int r_q)
{
    /*
     los indices de los nodos empieza desde 1;
     */
    if (l_q <= left_nodo && right_nodo <= r_q)
    {
        return st[nodo];
    }
    if (l_q > right_nodo || left_nodo > r_q)
    {
        return neutral();
    }

    int mitad = (left_nodo + right_nodo) / 2; ///[l:r] --> [l:mitad] [mitad+1:r]
    return merge(query(nodo * 2, left_nodo, mitad, l_q, r_q), query(nodo * 2 + 1, mitad + 1, right_nodo, l_q, r_q));
}

int main()
{
    cin >> n;
    int tam = n;
    int Q;
    cin >> Q;
    vector<ll> vec(n);
    for (int i = 0; i < n; i++)
    {
        cin >> vec[i];
    }

    while (__builtin_popcount(n) != 1)
    {
        vec.push_back(NEG_INF); // introduce neutro;
        n++;
    }
    st.resize(2 * n);

    buildSt(vec); // construimos el st

    for (int i = 1; i < 2 * n; i++)
    {
        if (__builtin_popcount(i) == 1)
            cout << endl;
        cout << st[i].sum << " ";
    }
    cout << endl;

    cout << query(1, 0, n - 1, 0, n - 1).ans << endl;

    while (Q--)
    {

        int indi;
        cin >> indi;
        int val;
        cin >> val;
        updateTreeNode(indi, make_data(val));
        for (int i = 1; i < 2 * n; i++)
        {
            if (__builtin_popcount(i) == 1)
                cout << endl;
            cout << st[i].ans << " ";
        }
        cout << endl;
        cout << st[1].ans << endl;
    }
    cout<<NEG_INF<<endl;

    return 0;
}\end{lstlisting}

\subsection*{minimo_range}
\begin{lstlisting}


#include <bits/stdc++.h>
using namespace std;
typedef vector<int> vi;

struct info
{
    int num;
};

info merge(info a, info b)
{
    info res;
    res.num = min(a.num, b.num);
    return res;
}

vector<info> st;
int n;

void buildSt(vi vec)
{
    for (int i = 0; i < n; i++) // Añadimos las hojas
    {
        info a;
        a.num = vec[i];
        st[n + i] = a;
    }
    for (int i = n - 1; i > 0; --i) // construmos los padres a partir de las hojas
    {
        st[i] = merge(st[i << 1], st[(i << 1) + 1]);
    }
}

void updateTreeNode(int p, info nuevoValor)
{
    // set value at position p
    st[p + n] = nuevoValor;
    p = p + n;

    // actualizar padres
    for (int i = p; i > 1; i >>= 1)
        st[i >> 1] = merge(st[i], st[i ^ 1]);
}

info query(int nodo, int left_nodo, int right_nodo, int l_q, int r_q)
{
    /*
     los indices de los nodos empieza desde 1;
     */
    if (l_q <= left_nodo && right_nodo <= r_q)
    {
        return st[nodo];
    }
    if (l_q > right_nodo || left_nodo > r_q)
    {
        info xd;
        xd.num = 1e9;
        return xd; // neutro
    }

    int mitad = (left_nodo + right_nodo) / 2; ///[l:r] --> [l:mitad] [mitad+1:r]
    return merge(query(nodo * 2, left_nodo, mitad, l_q, r_q), query(nodo * 2 + 1, mitad + 1, right_nodo, l_q, r_q));
}

int main()
{
    cin >> n;
    int Q;cin>>Q;
    vector<int> vec(n);
    for (int i = 0; i < n; i++)
    {
        cin >> vec[i];
    }

    while (__builtin_popcount(n) != 1)
    {
        vec.push_back(1e9); // introduce neutro;
        n++;
    }
    st.resize(2 * n);

    buildSt(vec); // construimos el st

    
    while(Q--)
    {
        int q;cin>>q;
        if(q==1){ //update, son 0 indexed

            int indi;cin>>indi;
            info val;cin>>val.num;
            updateTreeNode(indi,val);

        }
        else
        {
            int l,r;cin>>l>>r;
            //q(nodo inicio,rango inicio 0i,rango fin0i,q_inicio 0i,q_fin 0i)
            cout<<query(1,0,n-1,l,r-1).num<<endl;//la resta a l o r denpenda del input constrait

        }
    }

    

    return 0;
}\end{lstlisting}

\subsection*{sereja_and_brackets}
\begin{lstlisting}


#include <bits/stdc++.h>
using namespace std;
typedef vector<int> vi;
typedef long long ll;

struct info
{
    int a, b, c;
};

info merge(info x, info y)
{
    info ret;
    int t = min(x.b, y.c);
    ret.a = x.a + y.a + t;
    ret.b = x.b + y.b - t;
    ret.c = x.c + y.c - t;
    return ret;
}

info make_info(char ch) {
    if (ch == '(')
        return {0, 1, 0};
    else
        return {0, 0, 1};
}
info neutral() {
    return {0, 0, 0};
}

vector<info> st;
int n;

void buildSt(vector<char> vec)
{
    for (int i = 0; i < n; i++) // Añadimos las hojas
    {
        st[n + i] = make_info(vec[i]);
    }
    for (int i = n - 1; i > 0; --i) // construmos los padres a partir de las hojas
    {
        st[i] = merge(st[i << 1], st[(i << 1) + 1]);
    }
}

void updateTreeNode(int p, info nuevoValor)
{
    // set value at position p
    st[p + n] = nuevoValor;
    p = p + n;

    // actualizar padres
    for (int i = p; i > 1; i >>= 1)
        st[i >> 1] = merge(st[i], st[i ^ 1]);
}

info query(int nodo, int left_nodo, int right_nodo, int l_q, int r_q)
{
    /*
     los indices de los nodos empieza desde 1;
     */
    if (l_q <= left_nodo && right_nodo <= r_q)
    {
        return st[nodo];
    }
    if (l_q > right_nodo || left_nodo > r_q)
    {
        return neutral();
    }

    int mitad = (left_nodo + right_nodo) / 2; ///[l:r] --> [l:mitad] [mitad+1:r]
    return merge(query(nodo * 2, left_nodo, mitad, l_q, r_q), query(nodo * 2 + 1, mitad + 1, right_nodo, l_q, r_q));
}

int main()
{
    string cad;
    cin >> cad;
    n = cad.length();
    int Q;
    cin >> Q;
    bitset<10> xd;
    vector<char> vec(n);
    for (int i = 0; i < n; i++)
    {
        vec[i] = cad[i];
    }

    while (__builtin_popcount(n) != 1)
    {
        vec.push_back(0); // introduce neutro;
        n++;
    }
    st.resize(2 * n);

    buildSt(vec); // construimos el st

    // los updates deben ser 0 indexed

    while (Q--)
    {
        int l, r;
        cin >> l >> r;
        l--;
        r--;
        cout << query(1, 0, n - 1, l, r).a * 2 << endl;
    }

    return 0;
}\end{lstlisting}

\subsection*{sum}
\begin{lstlisting}


#include <bits/stdc++.h>
using namespace std;
typedef vector<int> vi;
typedef long long ll;

struct info
{
    ll num;
};

info merge(info a, info b)
{
    info res;
    res.num = a.num+b.num;
    return res;
}

vector<info> st;
int n;

void buildSt(vector<ll> vec)
{
    for (int i = 0; i < n; i++) // Añadimos las hojas
    {
        info a;
        a.num = vec[i];
        st[n + i] = a;
    }
    for (int i = n - 1; i > 0; --i) // construmos los padres a partir de las hojas
    {
        st[i] = merge(st[i << 1], st[(i << 1) + 1]);
    }
}

void updateTreeNode(int p, info nuevoValor)
{
    // set value at position p
    st[p + n] = nuevoValor;
    p = p + n;

    // actualizar padres
    for (int i = p; i > 1; i >>= 1)
        st[i >> 1] = merge(st[i], st[i ^ 1]);
}

info query(int nodo, int left_nodo, int right_nodo, int l_q, int r_q)
{
    /*
     los indices de los nodos empieza desde 1;
     */
    if (l_q <= left_nodo && right_nodo <= r_q)
    {
        return st[nodo];
    }
    if (l_q > right_nodo || left_nodo > r_q)
    {
        info xd;
        xd.num = 0;
        return xd; // neutro
    }

    int mitad = (left_nodo + right_nodo) / 2; ///[l:r] --> [l:mitad] [mitad+1:r]
    return merge(query(nodo * 2, left_nodo, mitad, l_q, r_q), query(nodo * 2 + 1, mitad + 1, right_nodo, l_q, r_q));
}

int main()
{
    cin >> n;
    int Q;
    cin >> Q;
    vector<ll> vec(n);
    for (int i = 0; i < n; i++)
    {
        cin >> vec[i];
    }

    while (__builtin_popcount(n) != 1)
    {
        vec.push_back(0); // introduce neutro;
        n++;
    }
    st.resize(2 * n);

    buildSt(vec); // construimos el st

   // los updates deben ser 0 indexed
    
    while(Q--)
    {
        int q;cin>>q;
        if(q==1){ //update, son 0 indexed

            int indi;cin>>indi;
            info val;cin>>val.num;
            updateTreeNode(indi,val);

        }
        else
        {
            int l,r;cin>>l>>r;
            //q(nodo inicio,rango inicio 0i,rango fin0i,q_inicio 0i,q_fin 0i)
            cout<<query(1,0,n-1,l,r-1).num<<endl;//la resta a l o r denpenda del input constrait

        }
    }



    

    return 0;
}\end{lstlisting}

\section*{.Github}

\section*{.Github/Workflows}

\section*{Geometria Computacional}

\subsection*{convex_hull}
\begin{lstlisting}
#include <bits/stdc++.h>
using namespace std;

typedef long long ll;

#define CRISTIANO_RONALDO_GANO_35_COPAS \
	ios_base::sync_with_stdio(false);   \
	cin.tie(NULL);                      \
	cout.tie(nullptr);
#define hola cout << "hola" << endl;
#define YES cout << "YES" << endl;
#define NO cout << "NO" << endl;
#define printv(v)                      \
	for (int i = 0; i < v.size(); i++) \
	{                                  \
		cout << v[i] << " ";           \
	}                                  \
	cout << "\n";
#define printvii(v)                                       \
	for (int i = 0; i < v.size(); i++)                    \
	{                                                     \
		cout << v[i].first << " " << v[i].second << endl; \
	}                                                     \
	cout << endl;

#define RAYA cout << "----------------------------" << endl;

// #define debug1(A) if(debug) { cerr << "? " << #A << " : " << A << endl; }

const int MOD = 1000000007;
const int INFINITO_INT = INT_MAX;
#define dbg(x) cout << #x << " " << x << endl;

struct pt
{
	double x, y;
	bool operator==(pt const &t) const
	{
		return x == t.x && y == t.y;
	}
};

int orientation(pt a, pt b, pt c)
{
	double v = a.x * (b.y - c.y) + b.x * (c.y - a.y) + c.x * (a.y - b.y);
	if (v < 0)
		return -1; // clockwise
	if (v > 0)
		return +1; // counter-clockwise
	return 0;
}

bool cw(pt a, pt b, pt c, bool include_collinear)
{
	int o = orientation(a, b, c);
	return o < 0 || (include_collinear && o == 0);
}
bool collinear(pt a, pt b, pt c) { return orientation(a, b, c) == 0; }

void convex_hull(vector<pt> &a, bool include_collinear = false)
{
	pt p0 = *min_element(a.begin(), a.end(), [](pt a, pt b)
						 { return make_pair(a.y, a.x) < make_pair(b.y, b.x); });
	sort(a.begin(), a.end(), [&p0](const pt &a, const pt &b)
		 {
        int o = orientation(p0, a, b);
        if (o == 0)
            return (p0.x-a.x)*(p0.x-a.x) + (p0.y-a.y)*(p0.y-a.y)
                < (p0.x-b.x)*(p0.x-b.x) + (p0.y-b.y)*(p0.y-b.y);
        return o < 0; });
	if (include_collinear)
	{
		int i = (int)a.size() - 1;
		while (i >= 0 && collinear(p0, a[i], a.back()))
			i--;
		reverse(a.begin() + i + 1, a.end());
	}

	vector<pt> st;
	for (int i = 0; i < (int)a.size(); i++)
	{
		while (st.size() > 1 && !cw(st[st.size() - 2], st.back(), a[i], include_collinear))
			st.pop_back();
		st.push_back(a[i]);
	}

	if (include_collinear == false && st.size() == 2 && st[0] == st[1])
		st.pop_back();

	a = st;
}

vector<pt> a;

int main()
{
	CRISTIANO_RONALDO_GANO_35_COPAS;
	
	int n;cin>>n;
	a.resize(n);

	//leer como puntos

	convex_hull(a,true);//true incluye colinear, false no lo hace

}\end{lstlisting}

\section*{Primalidad}

\subsection*{Rabin_Miller}
\begin{lstlisting}
#include <bits/stdc++.h>

using namespace std;


using u64 = uint64_t;
using u128 = __uint128_t;

u64 binpower(u64 base, u64 e, u64 mod) {
    u64 result = 1;
    base %= mod;
    while (e) {
        if (e & 1)
            result = (u128)result * base % mod;
        base = (u128)base * base % mod;
        e >>= 1;
    }
    return result;
}

bool check_composite(u64 n, u64 a, u64 d, int s) {
    u64 x = binpower(a, d, n);
    if (x == 1 || x == n - 1)
        return false;
    for (int r = 1; r < s; r++) {
        x = (u128)x * x % n;
        if (x == n - 1)
            return false;
    }
    return true;
};

bool MillerRabin(u64 n, int iter=5) { // returns true if n is probably prime, else returns false.
    cout<<iter<<endl;
    if (n < 4)
        return n == 2 || n == 3;

    int s = 0;
    u64 d = n - 1;
    while ((d & 1) == 0) {
        d >>= 1;
        s++;
    }

    for (int i = 0; i < iter; i++) {
        int a = 2 + rand() % (n - 3);
        if (check_composite(n, a, d, s))
            return false;
    }
    return true;
}


int main()
{

    cout<<MillerRabin(100000001,30)<<endl;
    return 0;
}\end{lstlisting}

\subsection*{pollard_Rho}
\begin{lstlisting}
#include <iostream>
#include <cstdlib>
#include <cstdio>
#include <cmath>
#include <cassert>
#include <map>

using namespace std;

typedef long long ll;

#define forn(i, n) for (int i = 0; i < (int)(n); i++)
#define forsn(i, s, n) for (int i = int(s); i < (int)(n); i++)

// rabin miller

ll potlog(ll a, ll b, const ll M)
{
    ll res = 1;
    while (b)
    {
        if (b % 2)
            res = (__int128(res) * a) % M;
        a = (__int128(a) * a) % M;
        b /= 2;
    }
    return res;
}

bool primo(ll n)
{
    if (n < 2)
        return false;
    if (n == 2)
        return true;
    ll D = n - 1, S = 0;
    while (D % 2 == 0)
    {
        D /= 2;
        S++;
    }
    // n-1 = 2^S * D
    static const int STEPS = 16;
    forn(pasos, STEPS)
    {
        const ll A = 1 + rand() % (n - 1);
        ll M = potlog(A, D, n);
        if (M == 1 || M == (n - 1))
            goto next;
        forn(k, S - 1)
        {
            M = (__int128(M) * M) % n;
            if (M == (n - 1))
                goto next;
        }
        return false;
    next:;
    }
    return true;
}

// pollard's rho

ll mcd(ll a, ll b) { return (a == 0) ? b : mcd(b % a, a); }

ll factor(ll n)
{
    static ll A, B;
    A = 1 + rand() % (n - 1);
    B = 1 + rand() % (n - 1);
#define f(x) ((__int128(x) * (x + B)) % n + A)
    ll x = 2, y = 2, d = 1;
    while (d == 1 || d == -1)
    {
        x = f(x);
        y = f(f(y));
        d = mcd(x - y, n);
    }
    return abs(d);
}

map<ll, ll> fact;

void factorize(ll n)
{
    assert(n > 0);
    while (n > 1 && !primo(n))
    {
        ll f;
        do
        {
            f = factor(n);
        } while (f == n);
        n /= f;
        factorize(f);
        for (auto &it : fact)
            while (n % it.first == 0)
            {
                n /= it.first;
                it.second++;
            }
    }
    if (n > 1)
        fact[n]++;
}

int main()
{
    ll N;
    while (cin >> N && N)
    {
        fact.clear();
        factorize(N);
        for (const auto &it : fact)
            cout << it.first << "^" << it.second << " ";
        cout << endl;
    }
    return 0;
}
\end{lstlisting}

\section*{Graphs}

\subsection*{toposort}
\begin{lstlisting}
#include <bits/stdc++.h>
using namespace std;
int n; // number of vertices
vector<vector<int>> gf;
vector<bool> vis;
vector<int> ans;

void dfs(int v) {//0i
    vis[v] = true;
    for (int u : gf[v]) {
        if (!vis[u]) {
            dfs(u);
        }
    }
    ans.push_back(v);
}

void topological_sort() {
    vis.assign(n, false);
    ans.clear();
    for (int i = 0; i < n; ++i) {
        if (!vis[i]) {
            dfs(i);
        }
    }
    reverse(ans.begin(), ans.end());
}\end{lstlisting}

\section*{Overloads}

\subsection*{pq}
\begin{lstlisting}
struct cmp
{
    //mayor tiene prioridad
    bool operator()(const int& a, const int& b) const {
        return a<b;        
    }
};

//priority_queue<Data,vector<Data>,cmp> pq
\end{lstlisting}

\subsection*{set}
\begin{lstlisting}
struct cmpST
{
    //de menor a mayor
    bool operator()(const int &a,const int &b)
    {
        return a<b;
    }
};

//set<int,cmpST> st;
\end{lstlisting}

\section*{.Git}

\section*{.Git/Hooks}

\section*{.Git/Refs}

\section*{.Git/Refs/Tags}

\section*{.Git/Refs/Remotes}

\section*{.Git/Refs/Remotes/Origin}

\section*{.Git/Refs/Heads}

\section*{.Git/Info}

\section*{.Git/Objects}

\section*{.Git/Objects/96}

\section*{.Git/Objects/2F}

\section*{.Git/Objects/0F}

\section*{.Git/Objects/F1}

\section*{.Git/Objects/83}

\section*{.Git/Objects/F0}

\section*{.Git/Objects/84}

\section*{.Git/Objects/62}

\section*{.Git/Objects/0C}

\section*{.Git/Objects/5A}

\section*{.Git/Objects/15}

\section*{.Git/Objects/3E}

\section*{.Git/Objects/1A}

\section*{.Git/Objects/0A}

\section*{.Git/Objects/Aa}

\section*{.Git/Objects/A4}

\section*{.Git/Objects/Eb}

\section*{.Git/Objects/8B}

\section*{.Git/Objects/Ad}

\section*{.Git/Objects/B9}

\section*{.Git/Objects/49}

\section*{.Git/Objects/14}

\section*{.Git/Objects/A5}

\section*{.Git/Objects/Ac}

\section*{.Git/Objects/F7}

\section*{.Git/Objects/B0}

\section*{.Git/Objects/0B}

\section*{.Git/Objects/90}

\section*{.Git/Objects/4C}

\section*{.Git/Objects/68}

\section*{.Git/Objects/Info}

\section*{.Git/Objects/4F}

\section*{.Git/Objects/61}

\section*{.Git/Objects/13}

\section*{.Git/Objects/F8}

\section*{.Git/Objects/2E}

\section*{.Git/Objects/6C}

\section*{.Git/Objects/Fb}

\section*{.Git/Objects/39}

\section*{.Git/Objects/43}

\section*{.Git/Objects/70}

\section*{.Git/Objects/7A}

\section*{.Git/Objects/3C}

\section*{.Git/Objects/D4}

\section*{.Git/Objects/27}

\section*{.Git/Objects/C2}

\section*{.Git/Objects/2D}

\section*{.Git/Objects/A3}

\section*{.Git/Objects/47}

\section*{.Git/Objects/3D}

\section*{.Git/Objects/8C}

\section*{.Git/Objects/81}

\section*{.Git/Objects/D9}

\section*{.Git/Objects/73}

\section*{.Git/Objects/9D}

\section*{.Git/Objects/5C}

\section*{.Git/Objects/B3}

\section*{.Git/Objects/B2}

\section*{.Git/Objects/7D}

\section*{.Git/Objects/C6}

\section*{.Git/Objects/Fc}

\section*{.Git/Objects/12}

\section*{.Git/Objects/B5}

\section*{.Git/Objects/26}

\section*{.Git/Objects/9E}

\section*{.Git/Objects/E9}

\section*{.Git/Objects/57}

\section*{.Git/Objects/63}

\section*{.Git/Objects/99}

\section*{.Git/Objects/Ff}

\section*{.Git/Objects/31}

\section*{.Git/Objects/Bb}

\section*{.Git/Objects/72}

\section*{.Git/Objects/37}

\section*{.Git/Objects/28}

\section*{.Git/Objects/A1}

\section*{.Git/Objects/Pack}

\section*{.Git/Objects/71}

\section*{.Git/Objects/1D}

\section*{.Git/Objects/E6}

\section*{.Git/Objects/85}

\section*{.Git/Objects/94}

\section*{.Git/Objects/E5}

\section*{.Git/Objects/87}

\section*{.Git/Objects/09}

\section*{.Git/Objects/93}

\section*{.Git/Objects/A6}

\section*{.Git/Objects/B1}

\section*{.Git/Objects/4B}

\section*{.Git/Objects/7F}

\section*{.Git/Objects/88}

\section*{.Git/Objects/36}

\section*{.Git/Objects/9F}

\section*{.Git/Objects/9A}

\section*{.Git/Objects/C5}

\section*{.Git/Objects/05}

\section*{.Git/Logs}

\section*{.Git/Logs/Refs}

\section*{.Git/Logs/Refs/Remotes}

\section*{.Git/Logs/Refs/Remotes/Origin}

\section*{.Git/Logs/Refs/Heads}

\section*{.Git/Branches}

\section*{.Vscode}



\end{multicols*}

\end{document}
